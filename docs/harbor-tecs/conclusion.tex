 %%%%%%%%%%%%%%%%%%
 % conclusion
 %%%%%%%%%%%%%%%%%%
\section{Conclusion}
\label{sec:conclude}
%
In this paper, we have explored the challenges in providing memory
protection in resource constrained embedded sensor nodes.
%
We have designed the Harbor memory protection scheme that creates and
enforces protection domains within a limited address space.
%
We have implemented the Harbor primitives in software to isolate the
dynamically loadable modules and the kernel in the SOS operating
system.
%
Even though the embodiment of the scheme is specific to SOS, the
general approach is applicable elsewhere.
%
By partitioning the Harbor primitives into hardware and software
components, we have designed a low-cost Micro-Memory Protection Unit
(UMPU) for tiny embedded processors.
%
We have implemented UMPU for the AVR ATMEGA103 microcontroller.
%

There is a large space of protection architectures that can be
designed using Harbor components.
%
A detailed exploration of the design space with an evaluation of the
various overheads is necessary to pick the most appropriate operating
point for a given system.
%
We believe that a complete system for memory protection would require
combination of two or more software based approaches.
% 
For example, a system composed of ASVM with memory safe extensions
creates a flexible and safe environment for executing high level
scripts.
%
Of particular interest is the combination of static analysis with
Harbor's run-time checks.
%
Static analysis can eliminate a lot of run-time checks that currently
occur in Harbor.
%
However, this increases the complexity of the verifier which cannot be
implemented on the sensor nodes.
%
An alternative could be to implement the verifier on the microservers
in a tiered sensor network.
%
It may be feasible to implement the sandboxing tool on the
microserver in which case the verifier can be removed.
%
UMPU can also benefit from static analysis.
%
By extending the instruction set, we can incorporate a safe store
instruction that incurs an overhead of run-time check.
%
However, safe store instruction is only introduced where static
analysis cannot determine the validity of a store operation.
%
The exploration of the design space consisting of low cost hardware
accelerators, static analysis and binary transformation tools, and a
software run time for memory protection is very promising indeed.
%
We envision that the application of these techniques will create
robust software that would enable long term sensor network
deployments.

