 %%%%%%%%%%%%%%%%%%
 % conclusion
 %%%%%%%%%%%%%%%%%%
\section{Conclusion}
\label{sec:conclude}
%
Memory protection is an enabling technology for creating robust
software systems.
%
Protection mechanisms were introduced in mainframes in the sixties and
desktops in the eighties.
%
The complexity of the sensor software motivates the need for
protection in embedded systems.
%
The design constraints of the embedded domain motivate new
advancements in software and architecture technology to create robust
systems.
%
In this thesis, we have explored the challenges in providing memory
protection in resource constrained embedded sensor nodes.
%
We designed the Harbor memory protection scheme that creates and
enforces protection domains within a limited address space.
%
We implemented the Harbor primitives in software to isolate the
dynamically loadable modules and the kernel in the SOS operating
system.
%
Even though the embodiment of the scheme is specific to SOS, the
general approach is applicable elsewhere.
%
By partitioning the Harbor primitives into hardware and software
components, we designed a low-cost Micro-Memory Protection Unit
(UMPU) for tiny embedded processors.
%
We have implemented UMPU for the AVR ATMEGA103 microcontroller.
%

\subsection{Future Work}
\label{sec:verifydesignspace}
%
There is a large space of protection architectures that can be
designed using Harbor components.
%
A detailed exploration of the design space with an evaluation of the
various overheads is necessary to pick the most appropriate operating
point for a given system.
%
We believe that a complete system for memory protection would require
combination of two or more software based approaches.
% 
For example, a system composed of ASVM with memory safe extensions
creates a flexible and safe environment for executing high level
scripts.
%

The Harbor design also allows trading off execution overhead, and code size
increase, against the complexity of the verifier.
%
Performance and code size increase are directly proportional to the
number of operations that are sandboxed.
%
The current implementation sandboxes \textit{all} unsafe operations.
%
This severely penalizes performance and code size but reduces the
complexity of the verifier, which requires only a single pass over the
entire binary and maintains no additional state.
%
Static analysis on the binary can reduce the number of operations
sandboxed by adding single checks that safely protect a series of
potentially unsafe operations.
%
However, the safety of such a check is harder to verify using a simple
verifier.
%A single check can be used to protect a series of potentially unsafe operations.
%
%A complex verifier can permit direct execution of potentially unsafe operations (such as stores) that have been protected by earlier checks.
%
An interesting area of future work is to explore this design space to
develop a rewriter--verifier combination that consumes limited
resources but improve performance, and reduces code size, of sandboxed
binaries.


%Of particular interest is the combination of static analysis with
%Harbor's run-time checks.
%
%Static analysis can eliminate a lot of run-time checks that currently
%occur in Harbor.
%
%However, this increases the complexity of the verifier which cannot be
%implemented on the sensor nodes.
%
Another alternative is to implement the verifier on the microservers 
in a tiered sensor network.
%
The verified binary can be distributed to a mote cluster through a
secure protocol.
%

%UMPU can also benefit from static analysis.
%
%By extending the instruction set, we can incorporate a safe store
%instruction that incurs an overhead of run-time check.
%
%However, safe store instruction is only introduced where static
%analysis cannot determine the validity of a store operation.
%
%This involes an exploration of the design space consisting of low cost hardware
%accelerators, static analysis and binary transformation tools, and a
%software run time for memory protection.
%
We envision that the application of these techniques will create
robust software that would enable long term sensor network
deployments.

