%%%%%%%%%%%%%%%%%%%
% Related Work
%%%%%%%%%%%%%%%%%%%
\section{Related Work}
\label{sec:related}
%
This section reviews other systems that provide memory protection.
%
The systems are broadly classified based on their application
domains.
%
We focus first on sensor networks and then survey the more general
space of desktop and server computing systems.
%
%=====================================================================
\subsection{Sensor Networks}
%
As sensor networks are being envisioned for long-term deployments,
there is an emerging interest to address reliability as primary design
concern~\cite{tkernel06sensys,regehr06utos,dutta05ipsn}.
%-------------------------------------------------------------
\subsubsection{Naturalization - \emph{t-kernel}}
%
\emph{t-kernel}~\cite{tkernel06sensys}, a runtime for Mica motes, also
rewrites binaries to make them safe for execution.
%
Harbor and \emph{t-kernel} represent different points in the design space of
software-based protection mechanisms.
%
\emph{t-kernel} enforces a strong isolation boundary between the application and
the kernel.
%
Through a process called \emph{naturalization}, the application binary is
rewritten on the sensor node to guarantee that the \emph{t-kernel} can
always safely regain control of the processor, even, for example, in the
presence of application infinite loops that could otherwise hang the
system.
%
The rewritten application binary contains an entire TinyOS operating system
image, in contrast to Harbor, where a sensor node can protect multiple
modules from one another.
%


\emph{t-kernel} also implements software-based differentiated
virtual memory,
%
which translates the addresses for all memory accesses made by a program
into the heap segment.
%
The overhead of virtual memory is unpredictable and can be very
high in the event of a swap from external flash.
%
Harbor does not implement virtual memory, but does enforce isolation at a
finer granularity than \emph{t-kernel}.
%
In particular, Harbor can protect application modules from
one another.
%

Harbor does not address control flow isolation, except as required to
enforce memory isolation; in particular, it cannot force a buggy module
stuck in an infinite loop to relinquish control of the CPU.
%
\emph{t-kernel} requires external flash memory, whereas Harbor makes use of
on-chip flash memory only.
%
%-------------------------------------------------------------
\subsubsection{Safe OS Extensions - UTOS}
%
Safe TinyOS~\cite{regehr06utos} uses CCured~\cite{ccured02necula} and static
analysis techniques to provide memory safety to TinyOS applications.
%
CCured performs complex pointer analysis to mark pointers as safe or
unsafe.
%
Much driver code performs arbitrary typecasts that can cause CCured to
fail or conservatively mark pointers as unsafe, which introduces a
performance penalty as the CCured run-time performs bounds checks on
unsafe pointers during code execution.
%
CCured provides memory safety at a much finer granularity than Harbor.
%

The UTOS framework allows untrusted extensions to
safely interface with Safe TinyOS components.
%
UTOS extensions are made type-safe and memory-safe using CCured and a
backend service that copies buffers when they are exchanged between the
extension and the Safe TinyOS core.
%
Extensions are not allowed to interact with one another.
%
Harbor allows safe buffer transfers without copying, and allows extensions
to interact, but its current simple runtime check infrastructure introduces
overhead that CCured can sometimes avoid.
%

The UTOS backend service also mediates resource requests and prevents any
extension from starving other extensions in the system.
%
Harbor does not make any guarantees on fair resource allocation.
%
%-----------------------------------------------------------------
\subsubsection{Application Specific Virtual Machines - Mat\'e}
%
Application Specific Virtual Machines (ASVM) such as
Mat\'e~\cite{asvm05nsdi}~\cite{levis02mate} and
Agilla~\cite{agilla05ipsn} are domain specific interpreters that
execute high-level application scripts.
%
Both Harbor and ASVM provide memory safety to applications, albeit
through different mechanisms.
%
Harbor isolates applications from one another.
%
ASVM performs type and bounds checks on all memory accesses.
%
ASVM's checks are in some ways more stringent than Harbor's protection,
since they also prevent scripts from corrupting \emph{their own}
memory.
%

However, errors in ASVM's native code implementation might corrupt any
memory on the node.
%
Further, for efficiency, ASVMs are designed to be easily extensible to
customize the interpreters for a specific application domain.
%
The extensions can also be introduced dynamically~\cite{balani06dvm}.
%
The ASVM extensions implement powerful high level opcodes that perform
complex operations.
%
The extensions are implemented in non-type safe languages and can be
buggy.
%
ASVM can use Harbor's isolation to become more robust to errors in
extensions.


The trade-offs between ASVM and Harbor are evaluated in detail in
Section~\ref{sec:eval}.
%
Harbor has a significantly lower execution overhead compared to ASVM.
%
But the run-time checks introduced by Harbor increase the code-size.
%
%-----------------------------------------------------------------
\subsubsection{Type Safe Languages - Virgil}
%
Type safe languages provide memory safety at a fine granularity.
%
Type-safe languages for resource constrained microcontrolleres is an
emerging area of research.
%
Most of the software developed for embedded systems is written in
unsafe languages such as C (or even assembly for low-level drivers). 
%
Popular sensor network programming language NesC~\cite{gay03nesc},
contains minimal extensions to C (such as the \texttt{atomic} keyword)
to prevent race-conditions.
%
However, it does not address memory safety.


A common criticism of safe languages or unsafe languages retrofitted
with type information~\cite{ccured02necula} is their excessive CPU and
memory consumption.
%
This often makes them unsuitable for resource constrained sensor
nodes.
%
Virgil~\cite{titzer06virgil} is a new programming language that
attempts to address some of these concerns.
%
By explicitly separating the initialization time from run-time, Virgil
allows an application to build complex data structures during
compilation and then run directly on bare hardware without a virtual
machine on a language run-time.
%
The separation allows the entire program heap to be available at
compile time and enables new optimizations that reduce memory size.
%
Virgil does not support dynamic memory allocation.
%
Therefore, it is currently suited for building application specific
static operating system images such as TinyOS~\cite{levis05t2}.
%
An interesting area of future research is to explore safe languages
(such as Virgil) as a primary extensibility mechanism for dynamic
sensor operating systems such as SOS~\cite{ram05sos}.
%
SPIN project~\cite{spin95sosp} at Univ. of Washington has similar
goals for the desktop operating systems.

Finally, a safe language restricts an implementation to a single
language; it ignores a large base of existing code.
%
Harbor operates on compiled binaries and is therefore independent of
any programming language and is applicable to all the existing code
base.
%
Type safe languages require unsafe extensions to interface to
low-level hardware\footnote{Virgil extends type-safety to the
  hardware}, though these extensions could be used sparingly.
%
Another problem with type-safe languages is the size of the system
that needs to be trusted.
%
A complete language compiler and run-time consists of an optimizer
that uses complex analysis to improve run-time efficiency.
%
This is a large and complicated code base, all of which needs to be
trusted.
%
In contrast, for Harbor, only the run-time components and the simple
verifier needs to be trusted.
%
The size of this system is considerably smaller than the size of a
compiler.
%
% VM*~\cite{vmstar05sensys} is a virtual machine construction kit
% targets resource constrained sensor nodes and supports a subset of the
% JAVA programming language.
% %
% The VM* framework automatically generates an application specific
% virtual machine.
% %
% The virtual machine is optimized to include only parts of the
% interpreter and the run-time that are needed.
% %
% A native API exposes the OS and hardware services to the JAVA
% applications.
% %
% VM* addresses the development of applications in JAVA but the
% underlying operating system and device drivers are still written in
% non type-safe languages such as C or assembly.
% %
% JAVA applications in VM* are type-safe.
% %
% Unlike Harbor and UTOS, the VM* applications are not isolated from the
% operating system.
%
%Sympathy, a debugging framework, has focussed on developing
%network-level protocols to diagnose/localize
%problems~\cite{nithya05sympathy} 
%
%We can leverage Sympathy framework to send diagnostic information to
%the basestation whenever an exception is triggered in Harbor
%
%At present, hardware support in sensor nodes to achieve high
%dependability is absent, with reboot of the entire node being the
%most common approach~\cite{dutta05ipsn}.
%
%
%=========================================================================
\subsection{Software-based Techniques}
%
The design considerations for an embedded sensor system are vastly
different from a desktop/server system.
%
Still some of the features in Harbor have been motivated from the
large range of software-based memory protection techniques proposed
for desktop/server systems.
%
%------------------------------------------------------------
\subsubsection{Software-based Fault Isolation (SFI)}
%
Software-based Fault Isolation (SFI) is a general technique that
restricts the address range of stores, jumps and calls by modifying a
program binary.
%
The key challenge to SFI is to introduce the restrictions efficiently, and in
a manner that they cannot be bypassed by maliciously designed input
code.
%
SFI was originally proposed by Wahbe et. al.~\cite{wahbe93sfi} (called
``sandboxing'').
%
The pseudo-code for to sandbox an address is shown in
Figure~\ref{fig:sbxpseudocode}.
%

\begin{figure}
 \begin{tabbing}
  \texttt{dedi}\=\texttt{cated-reg} $\Leftarrow$ \texttt{target-reg \&
    and-mask-reg} \\
  \>\emph{Use dedicated register} \texttt{and-mask-reg} \emph{to clear
    segment identifier bits.}\\
  \texttt{dedicated-reg} $\Leftarrow$ \texttt{target-reg |
    segment-reg} \\
  \>\emph{Use dedicated register} \texttt{segment-reg} \emph{to set
    segment identifier bits.}\\
  \texttt{store instruction uses dedicated-reg}
 \end{tabbing}
 \caption{Assembly pseudo code to sandbox address in
   \texttt{target-addr}}
 \label{fig:sbxpseudocode}
\end{figure}
%

Sandboxing enforces static partitioning on an application's address
space to enable safe sharing of the address space by multiple
cooperative modules.
%
By choosing address space boundaries to be powers of 2, sandboxing
restrictions are efficiently introduced through bit-masks.
%
%Specifically, the target address of a store instruction is forced to belong to its domain.
%
Dedicated registers guarantee that checks cannot be bypassed by
jumping into the middle of a sandbox sequence.
%
In addition, Wahbe et. al. also designed a low-latency cross fault
domain communication mechanism.
%

Harbor strives to enforce similar restrictions as SFI; disallow
stores and jumps to addresses outside the module's domain.
%
However, the architectural limitations of the embedded processors
motivate new design approaches.
%
First, the absence of virtual memory in embedded processors limits the
total address space to the available physical memory.
%
The on-chip SRAM in AVR microcontroller~\cite{avrdatasheet} is a meagre 4 Kb.
%
Therefore, static partitioning of address space is infeasible; it
would lead to excessing internal fragmentation of the limited memory.
%
Second, all the software domains on an embedded processor share a
common run-time stack.
%
SFI implementations for desktop processors setup a separate stack for
each domain within their allocated address space.
%
Protecting the shared run-time stack is a design challenge for Harbor.
%
Third, the instruction set architectures of embedded processors have
limited capabilities.
%
For instance, in AVR, the load and store operations can use only three
pair of registers for addressing.
%
Therefore, it is often infeasible to set aside a dedicated register
pair for sandboxing.
%
Techniques different from SFI have to be employed to ensure that
Harbor checks are not circumvented.
%
Fourth, there is no support for privileged instructions in an embedded
processor.
%
Since embedded software runs directly ``on metal'', therefore Harbor
has to detect and disallow certain potentially unsafe opcodes such
as store to program memory.
%


Many variations of SFI have emerged since Wahbe's original
implementation of this technique for two RISC architectures, the MIPS
and Alpha.
%
In particular, the x86 implementations of SFI also face one similar
constraint as Harbor; the lack of sufficient registers in the
architecture to set aside a dedicated register.
%
MiSFIT~\cite{small97misfit}, an assembly language re-writer designed
to isolated faults in C++ code written for an extensible operating
system, uses a hash table of legal jump targets.
%
Control flow is re-directed through a check that ensures that jump
targets appear in the hash table.
%
This eliminates the need for dedicated registers.
%
PittSFIeld~\cite{pittsfield} divides the program memory into chunks of
16 bytes\footnote{Any size that is a power of 2 would suffice.} and
ensures through padding with \texttt{NOPS} that control flow does not
change except at an entry or exit of a chunk boundary.
%
The sandboxing sequence is inserted entirely within a chunk and
therefore cannot be circumvented.


\subsubsection{Control Flow Integrity (CFI)}
%
Control Flow Integrity (CFI)~\cite{cfi05msr} policy dictates that
software execution must follow the path of a \emph{Control Flow Graph}
(CFG) determined ahead of time.
%
CFI has more general goals than SFI that focuses solely on restricting a
program's jumps.
%
CFI is implemented by labelling each potential jump target by a
uniquely encoded \texttt{NOP} sequence.
%
All computed control transfers check for the appropriate tag before
transferring control.
%
XFI~\cite{xfi06osdi} builds on the CFI foundations to offer a
flexible, generalized and more efficient form of software-based fault
isolation (SFI).
%
Further, XFI uses a two-stack execution model to ensure verifiable
control flow integrity.
%The two-stack execution model used by Harbor to ensure module control flow
%integrity was motivated by XFI~\cite{xfi06osdi}, a high-performance variant
%of SFI.
%
XFI's scoped stack holds data accessible only in the static scope of
each function, including return addresses and most local variables.
%
A separate allocation stack stores data that may be shared within the
functions in a module.
%
The two-stack execution model used by Harbor for ensuring control flow
integrity within a module was motivated by XFI.
%
%--------------------------------------------------------------------
\subsubsection{Safe OS Extensions}
%
Our work is also related to several other efforts in the
desktop/server space to isolate kernel modules such as device drivers
either using hardware support (Nooks)~\cite{swift05nooks} or through type-safe
languages (SPIN)~\cite{spin95sosp} or software fault isolation
(ExoKernel)~\cite{exo97sosp}.
%
Nooks separates modules into lightweight protection domains by
managing separate page-tables for each module.
%
Our approach is similar to Nooks in that it maintains memory ownership
information, and relies on standard interfaces to control flow and
access of resources.
%
%=======================================================================
\subsection{Hardware-based Approaches}
%
Page-based virtual memory systems have become the dominant form of
memory management in the modern general-purpose computer systems.
%
The coarse granularity of page-based protection, its high performance
overhead and complexity does not make it a popular choice in the
low-end embedded microcontrollers.
%
We review the current state of art in hardware based memory protection
and compare it with UMPU, a hardware-assisted fault isolation system.
%
\subsubsection{Memory Protection Units (MPU)}
\label{sec:mpu}
%
Memory Protection Unit (MPU) is used to statically partition memory
and set individual protection attributes for each partition.
%
The partitions are contiguous segments within the address space
defined by a pair of base and bounds registers.
%
The most common protection attributes are shown in
Table~\ref{tab:armprotattr}.
%
\begin{table}[htdp]
\centering
\small{
\begin{tabular}{|c|l|}
	\hline
	Code & Mode Description\\
	\hline
        00 & No Access \\
        01 & Privileged Mode Access Only \\
        10 & Privileged Mode Full Access, User Mode Read Only \\
        11 & Full Access\\
	\hline
\end{tabular}}
\caption{ARM940T Protection Attributes Descriptions}
\label{tab:armprotattr}
\end{table}
%

The protection model of MPU is not suited for the embedded sensor
software running on low-end microcontrollers.
%
MPU defines only two protection domains viz. User-mode and Supervisor
mode.
%
This is sufficient for protecting the kernel from the applications but
not the applications from one another.
%
The static partitioning of address space into contiguous regions is
infeasible for the low-end microcontrollers with very limited memory
footprint.
%
Further, the number of partitions is also limited.
%

However, MPU has a lower memory footprint than UMPU because the
partitioning information can be stored in registers instead of
maintaining a memory map.
%
MPU introduces no performance overhead while UMPU incurs a single
clock cycle penalty for memory map accesses.
%
MPU's are used in embedded processors such as ARM
940T~\cite{arm940tds} and Infineon TC1775~\cite{inftc1775ds}.
%--------------------------------------------------------------------
\subsubsection{Mondriaan Memory Protection}
%
Mondriaan Memory Protection (MMP)~\cite{witchel-asplos02-mondrian}
creates hardware enforced protection domains within a single address
space.
%
MMP inspects memory accesses at the instruction level from within the
processor pipeline to provide word-level protection.
%
MMP is designed for high performance desktop architectures and it uses
fairly complex and expensive hardware extensions to reduce overhead of
monitoring all accesses.
%
MMP also maintains a table of access permissions within memory for
every address space.
%
However, the tables are significantly larger that Harbor memory map as
MMP protection is more flexible and fine-grained.
%
Further, as chip area and cost are not dominant concerns, MMP
implements an expensive Protection Lookaside Buffer (PLB) to reduce
the overhead of checks introduced during memory accesses.
%
%------------------------------------------------------------------
\subsubsection{Hardware Accelerated Checkers}
%
Divya et. al.~\cite{divya06ccured} propose a system that enhances the
CCured~\cite{ccured02necula} run-time checks by performing them in
hardware.
%
A group of \texttt{XCHECK} custom instructions are added to the
Xtensa~\cite{xtensads}, a customizable processor from Tensilica Inc.
%
The \texttt{XCHECK} instructions are inserted into the CCured modified
source code using a source transformation tool.
%
UMPU has a similar motivation; using hardware accelerators for
improving performance.
%
UMPU does not enforce fine-grained type-safety, it only focuses on
fault isolation.
%
%------------------------------------------------------------------
\subsection{Summary}
%
This section positions Harbor in the category of fault isolation for
resource constrained microcontrollers with limited address space.
%
Harbor isolation is more fine-grained than other approaches for sensor
networks.
%
The design of Harbor, though motivated by systems in the
desktop/server domains, introduces new techniques to address the
constraints of the embedded domain.

