%%%%%%%%%%%%%%%%%
% ABSTRACT
%%%%%%%%%%%%%%%%% 
\noindent
Many sensor nodes contain resource constrained microcontrollers where user
level applications, operating system components, and device drivers share
a single address space with no form of hardware memory protection.
%
Programming errors in one application can easily corrupt the state of
the operating system or other applications.
%
In this thesis, we propose \textit{Harbor}, a memory protection system that
prevents many forms of memory corruption.
%
We use fault isolation (``sandboxing'') to restrict application memory
accesses and control flow to protection domains within the address
space.
%
%Limited memory on sensor nodes precludes static partitioning of the
% address space into different domains.
%
A flexible and efficient \emph{memory map} data structure records
ownership and layout information for memory regions; writes are
validated using the memory map.
%
Control flow integrity is preserved by maintaining a \emph{safe stack}
that stores return addresses in a protected memory region.
%
Run-time checks validate computed control flow instructions.
%
Cross domain calls perform low-overhead control transfers between domains.
%
Checks are introduced by rewriting an application's compiled
binary. 
%
The sandboxed result is verified on the sensor node before it is admitted
for execution. 
%
Harbor's fault isolation properties depend only on the
correctness of this verifier and the Harbor runtime.
%
%Sensor nodes only need to trust the correctness of the verifier in
% the overall system.
%
We have implemented and tested Harbor on the SOS operating system.
%
Harbor detected and prevented memory corruption caused by programming
errors  in application modules that had been in use for several
months.
%
Harbor's overhead, though high, is less than that of application-specific
virtual machines, and reasonable for typical sensor workloads.

We partition the Harbor protection primitives into hardware and
software components to design a Micro Memory Protection Unit (UMPU)
for resource constrained embedded processors.
%
The hardware-software co-design approach offers the benefits of protection with a
minimal performance impact.
%
UMPU's area overhead is modest and can be easily accommodated without
increasing the die area of a chip.
%
UMPU extensions require no modifications to the instruction set of the embedded
processor thereby making it usable with the existing toolchains; a very
practical feature of the design.
