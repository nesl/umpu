%==============================================================
% ABSTRACT
%==============================================================
\noindent
Many embedded systems contain resource constrained microcontrollers where applications, operating system components and device drivers reside within a single address space with no form of memory protection.
%
Programming errors in one application can easily corrupt the state of the operating system and other applications on the microcontroller.
%
In this paper we propose a system that provides memory protection in tiny embedded processors.\footnote{8, 16 and 32-bit microcontrollers with limited resources}.
%
Our system consists of a software run-time working with minimal low-cost architectural extensions to the processor core that prevents corruption of state by buggy applications.
%
We restrict memory accesses and control flow of applications to \textit{protection domains} within the address space.
%
The software run-time consists of a \textit{Memory map}: a flexible and efficient data structure that records ownership and layout information of the entire address space.
%
Memory map checks are done for \texttt{store} instructions by hardware accelerators that significantly improve the performance of our system.
%
We preserve control flow integrity by maintaining a \textit{Safe stack} that stores return addresses in a protected memory region.
%
Cross domain function calls are redirected through a software based jump table.
%Domain switches within a single address space is done with the help of a cross domain linker tool that generates a software jump table.
%
Enhancements to the microcontroller \texttt{call} and \texttt{return} instructions use the jump table to track the current active domain.
%
We have implemented our scheme on a VHDL model of ATMEGA103 microcontroller.
%
Our evaluations show that embedded applications can enjoy the benefits of memory protection with minimal impact on performance and a modest increase in the area of the microcontroller.
