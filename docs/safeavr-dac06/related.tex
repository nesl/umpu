%=======================================================================
% RELATED WORK
%=======================================================================
\section{Related Work}
\label{sec:related}
%
Page-based virtual memory systems have become the dominant form of
memory management in the modern general-purpose computer systems.
%
While the process model of the virtual memory systems delivers protection for embedded applications, it also increases the overhead in memory consumption and processor performance.
%
The memory consumption increases due to the need to store address translation tables.
%
The processor performance is impacted because context switches have a high overhead; page tables have to be setup for the new context.
%
To improve performance of virtual memory, architectural features such as Translation Lookaside Buffers (TLBs) and virtual-mapped caches are used that further increase the area, cost and complexity of the chip.
%
For example, the addition of MMU and cache in an ARM7TDMI core~\cite{arm7tdmi} increases its area ten fold and its power consumption two fold. 
%
Therefore, current MMU designs will never be used in the low end price sensitive microcontrollers.

%
Memory protection units (MPU) provide hardware assisted protection in  embedded processors such as ARM
940T~\cite{arm940tds} and Infineon TC1775~\cite{inftc1775ds}.
%
MPU can statically partition memory and set individual protection attributes for each partition.
%
The partitions are contiguous segments within the address space defined by a pair of base and bounds registers.
%
The protection model of MPU is not suited for the complex embedded
software (such as operating systems) running on low-end microcontrollers.
%
MPU defines only two protection domains viz. User-mode and Supervisor
mode.
%
This is sufficient for protecting the kernel from the applications but
not the applications from one another.
%
The static partitioning of address space into contiguous regions is
infeasible for the low-end microcontrollers with very limited memory
footprint.
%
Further, the number of partitions is also limited.
%
However, MPU has a lower memory footprint than UMPU because the
partitioning information can be stored in registers instead of
maintaining a memory map.
%
MPU introduces no performance overhead while UMPU incurs a single
clock cycle penalty for memory map accesses.
%


%
Mondrian Memory Protection (MMP)~\cite{witchel-asplos02-mondrian} inspects memory accesses at the instruction level from within the processor pipeline to provide word-level protection.
%
It uses fairly complex and expensive hardware extensions to reduce overhead of monitoring all accesses.
%
SafeMem~\cite{qin-hpca05-safemem} exploits existing ECC memory protection to guard memory regions and detect any illegal accesses through ECC violations.
%
However, these techniques require significant resources to be performed on tiny embedded processors.

%
Hardware support for memory safe execution of embedded software was recently proposed in~\cite{divya06ccured}.
%
This technique uses CCured~\cite{ccured02necula}, a tool that generates type safe C programs through pointer inference techniques.
%
Extensions to the instruction set architecture speed up the run-time bounds checking operations performed by CCured.
%
Our techniques apply directly to machine instructions and are therefore agnostic to programming languages.
%
Also, our hardware extensions do not modify the processor instruction set architecture.
%
Hence, we can continue to use existing compilers.
%
Custom modifications to compilers can become a source of new bugs.
%

Many software based approaches for memory protection have been proposed.
%
Type-safe languages such as Virgil~\cite{titzer06virgil} can flag illegal accesses at compile or run-time.
%
They provide fine-grained memory protection of individual objects.
%
Type-safe languages do not interface with code written in non type-safe languages.
%
However, most of the software developed for embedded systems is written in unsafe languages such as C (or even assembly for low-level drivers). 
%
Popular programming language NesC~\cite{gay03nesc}, contains minimal extensions to C (such as the \texttt{atomic} keyword) to prevent race-conditions that can cause memory corruption.
%
ASVM~\cite{asvm05nsdi} can also be used for providing memory protection. 
%
Software-based fault isolation for embedded processors has been proposed in ~\cite{ram07harbor}.
%
All the software based approaches have a significantly higher overhead than custom hardware extensions.
